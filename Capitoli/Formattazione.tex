\chapter{Formattazione}\label{chapter:formattazione}
Prima di introdurre il capitolo si può scrivere una breve introduzione su ciò che si andrà ad affrontare.
In questo capitolo, per esempio, saranno presentati alcuni esempi di formattazione degli oggetti più comuni di una Tesi in informatica.


\section{Capitoli, Sezioni e Sottosezioni}\label{sec:cap_sec_subsec}
Capitoli, sezioni e sottosezioni devono essere usate appropriatamente e non sostituire elenchi puntanti.
In particolare, i Capitoli devono riguardare macro-argomenti della tesi; ad esempio \textbf{Background}, \textbf{Obiettivi}, \textbf{Progettazione/Implementazione} e \textbf{Risultati}.
Questi capitoli sono semplicemente una traccia bisogna adattare a seconda delle esigenze.

Le sezioni invece devono riguardare argomenti all'interno della macro-area definita dal capitolo.
Ad esempio se più tecniche sono state utilizzate durante la tesi si può suddividere il capitolo \textbf{Progettazione/Implementazione} in sezioni, ognuna riguardante una delle tecniche sperimentate.

Le sottosezioni infine si possono utilizzare per descrivere concetti distinti all'interno di ogni sezione, ognuno dei quali deve avere una sua identità.
Ogni sottosezione deve avere un motivo per essere definita e non semplicemente separare parti di uno stesso discorso, per quello c'è l'indentazione (doppio ``a capo'' per separare parti distinte dello stesso paragrafo e ``$\backslash\backslash$'' per separare due paragrafi).

\subsection{Esempio di sottosezione}\label{subsec:es_subsec}
\blindtext

\section{Esempi di Immagini}\label{sec:images}
Di seguito alcuni esempi di immagini.
Figura~\ref{fig:one} presenta una singola immagine con ancoraggio ``\textbf{ht}" che chiede al altex di lasciare la figura dove si trova, se possibile, e altrimenti di metterla in alto alla pagina.
Figura~\ref{fig:two} presenta una singola immagine con due sotto-figure con ancoraggio ``\textbf{ht}".
Figura~\ref{fig:three} presenta una singola immagine con tre sotto-figure con ancoraggio ``\textbf{H}" che forza l'immagine nel posto scelto dall'utente, questa opzione è sconsigliata a meno di necessità particolari.
Ogni volta che un'immagine è inserita è buona norma citarla nel testo, altrimenti l'immagine non avrebbe un chiaro significato.
Nel caso delle sotto-immagini si possono citare anche loro, ad esempio Figura~\ref{fig:lev1} è composta da altre due sotto-figure.

\section{Esempi di Codice e Misc.}\label{sec:code}
Alcuni esempi di codice, definizioni, tabelle e altro.

\begin{algorithm}[ht]
	\caption{Esempio di pseudo-codice}
	\label{alg:Prim_Mst}
	\begin{algorithmic}[1]
		\Statex
		\Function{MST-Prim}{grafo G, funzione\_peso $\omega$, nodo\_radice r}\\
		\State{\textit{Q}: coda di priorita contenente tutti i vertici in \textit{V}}
		\For{ogni \textit{u} $\in$ V(G)}
		\Let{\textit{u.key}}{$\infty$}
		\Let{\textit{u.}$\pi$}{\textit{NIL}} 
		\EndFor
		\Let{\textit{r.key}}{0}
		\Let{\textit{Q}}{\textit{V(G)}}
		\While{\textit{Q} $\neq$ 0}
		\Let{\textit{u}}{EXTRACT-MIN(\textit{Q})}
		\For{ogni \textit{v} $\in$ \textit{G.Adj[u]}}
		\If{$v \in Q$ and $\omega(u,v) < v.key$}
		\Let{\textit{v.key}}{$\omega(u,v)$}
		\Let{\textit{v.}$\pi$}{\textit{v}}
		\EndIf
		\EndFor
		\EndWhile
		\State \Return{}
		\EndFunction
	\end{algorithmic}
\end{algorithm}


\begin{lstlisting}[caption={Esempio di definizione struttura dati C++, ma non algoritmo},
	label={lst:block_struct}]
	struct Block
	{
		int id_block;
		char resolution;
		int id_subdomain;
		int key;
		bool in_other_subdomain;
		struct neigh_t neighbors[4];
	};
\end{lstlisting}



\begin{algorithm}[ht]
	\caption{Esempio di codice imperativo}
	\label{alg:my_prim_multi}
	\begin{algorithmic}[1]
		\Statex
		\Function{PRIM\_MULTI-MST}{roots\_list \textit{roots}, array\_Block \textit{adjacency\_list}}
		
		\State{\color{blue}{//Inizializzazione di tutti i vertici}}
		\For{ogni \textit{u} $\in$ \textit{adjacency\_list}}
		\Let{\textit{u.key}}{$\infty$}
		\Let{\textit{u.id\_subdomains}}{\textit{NIL}}
		\Let{\textit{u.in\_other\_subdomain}}{FALSE} 
		\EndFor
		
		\State{\color{blue}{//Inizializzazione di tutte le radici}}
		\Let{\textit{i}}{0}
		\For{ogni \textit{r} $\in$ \textit{roots}}
		\Let{\textit{r.key}}{0}
		\Let{\textit{r.id\_subdomains}}{\textit{i}}
		\Let{\textit{i}}{$i+1$}
		\EndFor
		\State{\textit{Q}: coda di priorita ordinata in base al campo 
			\textit{key}}
		\Let{\textit{Q}}{\textit{roots}}
		\Let{\textit{count}}{0}
		
		\While{count $<$ \textit{adjacency\_list.size}}
		\Let{\textit{u}}{EXTRACT-MIN(\textit{Q})}
		
		\If{!(\textit{u.in\_other\_subdomain})}
		\State{\color{blue}{//Essendo il minimo viene aggiunto definitivamente}}
		\Let{\textit{u.in\_other\_subdomain}}{TRUE}
		\State{\color{blue}{//Si analizzano i vicini}}
		\For{ogni \textit{v} $\in$ \textit{u.neighbors}}
		\State{\color{blue}{//In questo caso specifico il peso di ogni arco vale 1}}
		\If{$v \in adjacency\_list$ and ($u.key+1) < v.key$}
		\Let{\textit{v.key}}{$u.key+1$}
		\Let{\textit{v.subdomains}}{\textit{u.subdomains}}
		\Let{\textit{count}}{\textit{count} + 1}
		\EndIf
		\EndFor
		\EndIf
		\EndWhile
		\State \Return{}
		\EndFunction
	\end{algorithmic}
\end{algorithm}

\begin{algorithm}[ht]
	\caption{esempio di algoritmo in C++}
	\label{lst:genic_mpi}
	\begin{lstlisting}
		#include (*@\textquotedblleft@*)mpi.h"
		
		(*@~ ~ ~ ~\raisebox{-1pt}[0pt][0pt]{$\vdots$}@*)
		
		main(int argc, char** argv){
			
			(*@~ ~ ~ ~ ~ ~ ~ ~ ~ ~{\raisebox{-1pt}[0pt][0pt]{$\vdots$}}@*)
			
			//Nessuna chiamata a funzioni MPI prima di questa
			MPI_Init(&argc, &argv);
			
			(*@~ ~ ~ ~ ~ ~ ~ ~ ~ ~{\raisebox{-1pt}[0pt][0pt]{$\vdots$}}@*)
			
			MPI_Finalize();
			//Nessuna chiamata a funzioni MPI dopo questa
			
			(*@~ ~ ~ ~ ~ ~ ~ ~ ~ ~{\raisebox{-1pt}[0pt][0pt]{$\vdots$}}@*)
		}	
	\end{lstlisting}
\end{algorithm}

\begin{defn}[Esempio di Definizione]
	Un piccolo esempio di definizione
\end{defn}

\begin{table}[ht]
	\centering
	\resizebox{0.99\linewidth}{!}{
		\begin{tabular}{||c||c|c|c|c|c|c|c|c|c|c||}
			\hhline{|t:=:t:==========:t|}
			CT scan &\phantom{00}01\phantom{00}&\phantom{00}02\phantom{00}&\phantom{00}03\phantom{00}&\phantom{00}04\phantom{00}&\phantom{00}05&\phantom{00}06\phantom{00}&\phantom{00}07\phantom{00}&\phantom{00}08\phantom{00}&\phantom{00}09\phantom{00}&\phantom{00}10\phantom{00} \\
			\hhline{|:=::==========:|}
			Input arteries (n.) &66&175&76&134&198&172&154&108&91&65 \\
			\hhline{||-||----------||}
			Labeling time (s)&19.13&87.38&26.11&47.08&90.89&356.95&303.77&91.95&16.01&21.88 \\
			\hhline{||-||----------||}
			Grounding time (s)&9.76&52.45&9.57&22.88&70.34&47.52&37.44&17.98&10.61&8.04 \\
			\hhline{||-||----------||}
			Optimum time (s)&6.66&31.05&14.67&20.63&15.2&29.17&54.71&61.03&4.28&7.24 \\
			\hhline{|b:=:b:==========:b|}
		\end{tabular}
	}
	\vspace*{2mm}
	\caption{Esempio di tabella}
	\label{tab:perf}
\end{table}


\section{Citazioni}
Durante la scrittura della Tesi è necessario fornire i riferimenti bibliografici che hanno permesso la stesura della tesi stessa.
Questo si ottiene semplicemente tramite l'utilizzo del comando ``cite'' che prende come argomento la label di una delle entry nel file .bib.
Quindi per citare una qualunque fonte, ad esempio ``Artificial Intelligence - A Modern Approach", si può fare \cite{modernApproach} ($\backslash\text{cite}\{\text{modernApproach}\}$).
Questo leggerà la referenza nel file .bib che ha come label ``cormen".
Per citare più riferimenti contemporaneamente basta sperare le varie labels con una virgola come segue \cite{gelfond1998action,modernApproach,durfee1999distributed,de2003resource,allen2009complexity,bernstein2002complexity}.
Le reference non citate non appariranno in bibliografia.
\href{https://dblp.uni-trier.de/}{dblp} e \href{https://scholar.google.com/}{Google Scholar} sono siti nei quali estrarre la reference per bibtex per un lavoro di cui si conosce solo il titolo, ad esempio.