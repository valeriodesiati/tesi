\chapter*{Conclusione} %Se si cambia il Titolo cambiare anche la riga successiva così che appia corretto nell'conclusione
\addcontentsline{toc}{chapter}{Conclusione} %Per far apparire Introduzione nell'indice (Il nome deve rispecchiare quello del chapter)
Al termine della realizzazione del progetto di stage tutti gli obiettivi descritti nel paragrafo \ref{sec:obiettivi} sono stati raggiunti nei tempi prefissati, 
secondo la ripartizione oraria riportata al paragrafo \ref{sec:ripartizionelavoro}.\\ \\

Il progetto, nella sua fase iniziale, non è risultato di semplice realizzazione in quanto è stato richiesto l'utilizzo di tecnologie e strumenti di cui non si era a conoscenza.\\ \\

Nel corso dello sviluppo ci sono state diverse problematiche:
\begin{itemize}
\item Immagini container Docker.
\item Specifiche della Virtual Machine utilizzata.
\end{itemize}
\  \\
All'inizio si era provato ad utilizzare l'immagine Docker \texttt{ubuntu:latest}, per poi decidere, in modo definitivo, di utilizzare l'immagine ufficiale di Kong Gateway 
per problemi di compatibilità e di raggiungibilità del Gateway.

Ancora, come spiegato nei capitoli precedenti, l'applicazione si trova all'interno di una Virtual Machine sulla quale è stato installato Docker con il relativo container Kong.\\
La problematica era relativa al fatto che in momenti completamente casuali l'applicazione risultava irraggiungibile, e quindi inutilizzabile.\\ 
Andando ad analizzare le statistiche di utilizzo su Microsoft Azure ci si è resi conto che il problema risiedeva nelle risorse (molto limitate) della Virtual Machine, 
che quindi accusava problemi di eccessivo utilizzo di CPU e RAM che implicava l'impossibilità di elaborare le richieste ricevute.\\ \\ 

Quindi, parlando di eventuali sviluppi futuri del progetto, sicuramente rientra tra questi il miglioramento delle specifiche della Virtual Machine da utilizzare, 
tenendo conto del fatto che le risorse a disposizione con l'attuale Virtual Machine sono 1 CPU e 512 MB di memoria RAM.\\
Altri sviluppi futuri potrebbero concentrarsi sull'implementazione di un Database più ampio e ottimizzato, andando ad utilizzare ad esempio Microsoft Azure SQL Database, 
il Database integrato nella piattaforma Azure.
