\chapter*{Introduzione} %Se si cambia il Titolo cambiare anche la riga successiva così che appia corretto nell'indice
\addcontentsline{toc}{chapter}{Introduzione} %Per far apparire Introduzione nell'indice (Il nome deve rispecchiare quello del chapter)
\pagenumbering{arabic} % Settaggio numerazione normale
% L'introduzione deve contenere un riassunto del lavoro di Tesi.
% In particolare bisogna esprimere chiaramente e molto sinteticamente: contesto dello studio, motivazioni, contributo e conclusioni.
% Bisogna quindi fare un sommario dello studio ad alto livello, fornendo le intuizioni senza ricadere in dettagli tecnici.
% Anche lo stile dovrebbe essere più discorsivo rispetto alle parti tecniche della tesi.

Prima di addentrarsi nelle specifiche di questo progetto è bene partire dal concetto di applicazioni cloud – native.\\
Un'applicazione cloud – native è un'applicazione concepita e realizzata per risiedere in cloud.\\
L'approccio da utilizzare per lo sviluppo di questa tipologia di applicazioni è diametralmente opposto a quello per lo sviluppo di un'applicazione monolitica, 
come sarà spiegato in seguito nel paragrafo \ref{sec:microserviziintro}.\\ \\
Le applicazioni cloud – native si basano su tre concetti fondamentali:
\begin{itemize}
\item Orchestratori di container;
\item Microservizi;
\item Scalabilità.
\end{itemize}

Lo scopo di questo progetto è stato di realizzare un plugin per l'API Gateway Kong.\\
Il plugin prevede l'integrazione di Kong con un microservizio per verificare l'avvenuto acquisto di un modulo applicativo da parte di un utente, identificato tramite token.
Il progetto includerà la realizzazione del microservizio e l'ottimizzazione del plugin tramite l'utilizzo di una cache locale a Kong.\\
Il plugin, insieme al Gateway, funge quindi da tramite tra l'utente e il microservizio, intercettando tutte le richieste dell'utente, analizzandole, scomponendole e 
inoltrandole al microservizio, che risponderà al plugin che a sua volta inoltrerà la risposta all'utente.\\ \\
Quindi, il progetto ha tre macro – componenti:
\begin{itemize}
\item Kong Gateway;
\item Plugin per Kong Gateway scritto in linguaggio Lua;
\item Microservizio scritto in linguaggio Java.
\end{itemize}

L'obiettivo del progetto è quello di realizzare un software che consenta l'autenticazione di un utente all'interno di una piattaforma generica tramite un token 
da includere all'interno della richiesta inviata, senza quindi l'utilizzo di password o altri strumenti di autenticazione, 
come dettagliato nel \emph{Capitolo \ref{chap:requisiti} – Requisiti e casi d'uso}. \\

Si è deciso, insieme al Tutor Aziendale, di puntare su Kong Gateway come API Gateway per questo progetto proprio perché consente l'installazione di plugin 
proprietari e custom, oltre a fornire tutte le funzioni di base di un API Gateway.\\
Per quanto riguarda lo sviluppo del plugin, si è voluto utilizzare il linguaggio Lua per le sue caratteristiche di semplicità di utilizzo e integrazione e leggerezza.\\ \\

Per la realizzazione del microservizio è stato deciso di utilizzare il linguaggio Java e il framework Spring perché sono stati reputati più adatti allo scopo.\\
Il framework Spring fornisce delle caratteristiche che risultano essere comode e sicure per il programmatore per quanto riguarda la creazione di microservizi, 
gestione dei dati, integrazione con un Database ecc., il tutto mantenendo intatti i costrutti e le pratiche del linguaggio Java.\\
Tutte le tecnologie utilizzate all'interno del progetto sono approfondite nel \emph{Capitolo \ref{chap:tecnologie} – Tecnologie utilizzate}.\\ \\

Più nello specifico, la realizzazione del progetto si è articolata in diverse fasi, iniziando dallo sviluppo del Database e del microservizio in Java.\\
Lo sviluppo è stato effettuato all'inizio in locale, con i relativi test di corretto funzionamento e, successivamente, si è passati all'installazione di Kong Gateway
sulla Virtual Machine creata su Microsoft Azure e quindi allo sviluppo del plugin in Lua per il Gateway.\\
Infine, si è posto il microservizio all'interno della Virtual Machine (con il relativo ambiente di esecuzione, la JVM) e quindi sono state finalizzate tutte le 
impostazioni per la corretta comunicazione tra microservizio, plugin, Gateway e utente.\\
Per la fase di test, sia in locale che in remoto, è stato utilizzato il software Postman.\\
Tutti questi argomenti, lo sviluppo e il testing, saranno approfonditi rispettivamente nel \emph{Capitolo \ref{chap:progettazionesviluppo} – Progettazione e sviluppo} 
e \emph{Capitolo \ref{chap:testing} – Testing}.

