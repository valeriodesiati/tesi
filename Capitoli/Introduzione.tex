\chapter*{Introduzione} %Se si cambia il Titolo cambiare anche la riga successiva così che appia corretto nell'indice
\addcontentsline{toc}{chapter}{Introduzione} %Per far apparire Introduzione nell'indice (Il nome deve rispecchiare quello del chapter)
\pagenumbering{arabic} % Settaggio numerazione normale
% L'introduzione deve contenere un riassunto del lavoro di Tesi.
% In particolare bisogna esprimere chiaramente e molto sinteticamente: contesto dello studio, motivazioni, contributo e conclusioni.
% Bisogna quindi fare un sommario dello studio ad alto livello, fornendo le intuizioni senza ricadere in dettagli tecnici.
% Anche lo stile dovrebbe essere più discorsivo rispetto alle parti tecniche della tesi.

Prima di addentrarsi nelle specifiche di questo progetto è bene partire dal concetto di applicazioni cloud – native.\\
Un'applicazione cloud – native è un'applicazione concepita e realizzata per risiedere in cloud.\\
L'approccio da utilizzare per lo sviluppo di questa tipologia di applicazioni è diametralmente opposto a quello per lo sviluppo di un'applicazione monolitica, 
come sarà spiegato in seguito nel paragrafo \ref{sec:microserviziintro}.\\
\\
Le applicazioni cloud – native si basano su tre concetti fondamentali:
\begin{itemize}
\item Orchestratori di container
\item Microservizi
\item Scalabilità
\end{itemize}

Proprio nel rispetto di questa filosofia, sono stati introdotti i cosiddetti \emph{Orchestratori di container}, che sono utilizzati per racchiudere e allo stesso tempo 
isolare, una o più applicazioni nel loro ambiente di esecuzione, con i relativi file necessari.\\ \\
In altre parole, i container possono essere visti come delle Virtual Machine (infatti possiedono molte delle caratteristiche) che però non eseguono un sistema operativo 
nella sua interezza, ma solo l'applicazione di cui si necessita ed eventuali altri servizi coinvolti nel ciclo di vita di questa.
I container risultano essere comodi e affidabili per l'utilizzo in tutti i momenti dello sviluppo di un nuovo software, dallo sviluppo, al test, 
ino alla fase finale di produzione.\\ Altri due vantaggi che si possono ottenere dall'utilizzo dei container, così come per le Virtual Machine, 
riguardano la sicurezza e la scalabilità.\\
La prima è conseguenza del fatto che, come detto sopra, si riesce ad isolare l'applicazione in esecuzione in un determinato container e quindi, se dovessero esserci 
problemi (nella maggior parte dei casi possono verificarsi problemi inevitabili), questi non potrebbero intaccare in nessun modo altri container e quindi il funzionamento 
di altre applicazioni.\\
La scalabilità è garantita dal fatto che un container può essere creato, gestito e modificato in base alle necessità, non esistono \emph{container standard}, ognuno 
viene adattato per lo scopo da raggiungere.

I microservizi possono essere definiti come la scomposizione di applicazioni in elementi più piccoli, così da ottenere diversi vantaggi, come:
\begin{itemize}
\item \textbf{Scalabilità}\\ C'è la possibilità di misurare il carico di lavoro di un singolo servizio in ogni momento e adattarlo di conseguenza (si pensi alla differenza 
che può esserci in termini di carico di lavoro, ad esempio, nelle ore diurne e nelle ore notturne) il che può portare diversi vantaggi.
\item \textbf{Semplicità di distribuzione}\\ I microservizi possono essere distribuiti con l'approccio \emph{CI/CD} (Continuous Integration/Continuous Delivery), 
proprio come è stato fatto nella realizzazione di questo progetto, così da semplificare l'integrazione e il testing di nuove funzionalità.
\item \textbf{Indipendenza}\\ Si possono gestire gli errori di un determinato servizio isolandolo, senza bloccare l'intera applicazione.
\end{itemize}

Lo scopo di questo progetto è stato di realizzare un plugin per l'API Gateway Kong.\\
Il plugin prevede l'integrazione di Kong con un microservizio per verificare l'avvenuto acquisto di un modulo applicativo da parte di un utente, identificato tramite token.
Il progetto includerà la realizzazione del microservizio e l'ottimizzazione del plugin tramite l'utilizzo di una cache locale a Kong.\\
Il plugin, insieme al Gateway, funge quindi da tramite tra l'utente e il microservizio, intercettando tutte le richieste dell'utente, analizzandole, scomponendole e 
inoltrandole al microservizio, che risponderà al plugin che a sua volta inoltrerà la risposta all'utente.\\ \\
Quindi, il progetto ha tre macro – componenti:
\begin{itemize}
\item Kong Gateway
\item Plugin per Kong Gateway scritto in linguaggio Lua
\item Microservizio scritto in linguaggio Java
\end{itemize}

Si è deciso, insieme al Tutor Aziendale, di puntare su Kong Gateway come API Gateway per questo progetto proprio perché questo consente l'installazione di plugin 
proprietari e custom, oltre a fornire tutte le funzioni di base di un API Gateway.\\ \\
Per quanto riguarda lo sviluppo del plugin, si è voluto utilizzare il linguaggio Lua per le sue caratteristiche di semplicità di utilizzo e integrazione e leggerezza.\\ \\

Infine, per la realizzazione del microservizio è stato deciso di utilizzare il linguaggio Java e il framework Spring perché sono stati reputati più adatti allo scopo.\\
Il framework Spring fornisce delle caratteristiche che risultano essere comode e sicure per il programmatore per quanto riguarda la creazione di microservizi, 
gestione dei dati, integrazione con un Database ecc., il tutto mantenendo intatti i costrutti e le pratiche del linguaggio Java.
\\ \\
Tutte le caratteristiche e le specifiche tecniche riguardanti le macro – componenti e le micro – componenti del progetto saranno esposte in seguito.

