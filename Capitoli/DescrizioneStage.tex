\chapter{Descrizione dello stage}\label{chapter:formattazione}


\section{Progetto di stage}\label{sec:cap_sec_subsec}


\section{Azienda}\label{sec:images}
AESYS S.r.l. è un’azienda fondata nel 2013 a Pescara (PE) che si occupa di sviluppo software e consulting con sedi a Pescara e Torino. Ad oggi AESYS conta più di 240 dipendenti.\\
Nel corso degli anni l’azienda ha vissuto una grande crescita e ora dispone di figure specializzate in molteplici campi.\\

\subsection{Servizi offerti}
AESYS è attiva in molti campi dell’ambito dell’\emph{Information Technology} fornendo servizi per piattaforme Web e Mobile, in ambito DevOps, Cloud, Machine Learning e sviluppo UX/UI.

\subsection{Organizzazione}
AESYS è suddivisa in \emph{Business Unit}, ognuna per campo di sviluppo.\\
In particolare, nella realizzazione di questo progetto sono state coinvolte due Business Unit: la RED, che si focalizza su tutte le tecnologie legate al linguaggio Java e alla JVM e la ORANGE, unità in ambito DevOps che fornisce supporto per manutenzione e monitoraggio dei sistemi su grandi infrastrutture.

\subsection{Metodologie aziendali}
AESYS adotta la metodologia di sviluppo Agile da diverso tempo, applicandone i principi nel quotidiano.\\
La metodologia di sviluppo Agile si basa su dei valori fondamentali:
\begin{itemize}
	\item[$\bullet$]Il personale e le interazioni sono più importanti dei processi e degli strumenti.
	\item[$\bullet$]È meglio avere un software funzionante che una documentazione esaustiva.
	\item[$\bullet$]La collaborazione con il cliente è più importante della stipula di un contratto.
	\item[$\bullet$]Essere pronti al cambiamento.
\end{itemize}

\subsubsection{Strumenti di supporto}
In AESYS lo strumento più utilizzato per comunicare è Microsoft Teams.\\
Il software in questione permette di creare al proprio interno una vera e propria gerarchia aziendale, programmare meeting, avere agende condivise, instant messaging e tante altre risorse utili per lavorare in team.
All’interno dell’azienda viene utilizzato \emph{Git} come sistema di versionamento del software (argomento approfondito nel capitolo \emph{Tecnologie utilizzate}).\\

\begin{figure}[ht]
	\centering
	\resizebox{.3\textwidth}{!}{\includegraphics{img/aesys}}
	\caption{Logo di AESYS S.r.l.}
	\label{fig:one}
\end{figure}

\section{Introduzione al progetto}\label{sec:code}
\section{Obiettivi}