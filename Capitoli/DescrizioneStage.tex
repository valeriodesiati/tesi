\chapter{Descrizione dello stage}\label{chapter:descrizione}
In questo capitolo saranno esposte brevemente tutte le caratteristiche e le dinamiche dello stage svolto, che saranno approfondite nei capitoli successivi.

\section{Azienda}\label{sec:azienda}
AESYS S.r.l. è un’azienda fondata nel 2013 a Pescara (PE) che si occupa di sviluppo software e consulting con sedi a Pescara e Torino. Ad oggi AESYS conta più di 240 dipendenti.\\
Nel corso degli anni l’azienda ha vissuto una grande crescita e ora dispone di figure specializzate in molteplici campi.\\

\subsection{Servizi offerti}
AESYS è attiva in molti campi dell’ambito dell’\emph{Information Technology} fornendo servizi per piattaforme Web e Mobile, in ambito DevOps, Cloud, Machine Learning e sviluppo UX/UI.

\subsection{Organizzazione}
AESYS è suddivisa in \emph{Business Unit}, ognuna per campo di sviluppo.\\
In particolare, nella realizzazione di questo progetto sono state coinvolte due Business Unit: la RED, che si focalizza su tutte le tecnologie legate al linguaggio Java e alla JVM e la ORANGE, unità in ambito DevOps che fornisce supporto per manutenzione e monitoraggio dei sistemi su grandi infrastrutture.

\subsection{Metodologie aziendali}
AESYS adotta la metodologia di sviluppo Agile da diverso tempo, applicandone i principi nel quotidiano.\\
La metodologia di sviluppo Agile si basa su dei valori fondamentali:
\begin{itemize}
	\item[$\bullet$]Il personale e le interazioni sono più importanti dei processi e degli strumenti.
	\item[$\bullet$]È meglio avere un software funzionante che una documentazione esaustiva.
	\item[$\bullet$]La collaborazione con il cliente è più importante della stipula di un contratto.
	\item[$\bullet$]Essere pronti al cambiamento.
\end{itemize}

\subsubsection{Strumenti di supporto}
In AESYS lo strumento più utilizzato per comunicare è Microsoft Teams.\\
Il software in questione permette di creare al proprio interno una vera e propria gerarchia aziendale, programmare meeting, avere agende condivise, instant messaging e tante altre risorse utili per lavorare in team.
All’interno dell’azienda viene utilizzato \emph{Git} come sistema di versionamento del software (argomento approfondito nel capitolo \emph{Tecnologie utilizzate}).\\

\begin{figure}[ht]
	\centering
	\resizebox{.3\textwidth}{!}{\includegraphics{img/aesys}}
	\caption{Logo di AESYS S.r.l.}
	\label{fig:one}
\end{figure}

\section{Progetto di stage}\label{sec:progetto}
Il progetto di stage prevede la realizzazione di un plugin per \emph{Kong Gateway} scritto con il linguaggio \emph{Lua}.\\
Il plugin prevede l'integrazione del gateway con un microservizio scritto in Java per verificare l'avvenuto acquisto di un modulo applicativo da parte di un utente identificato tramite token.\\
Il progetto comprende la realizzazione del microservizio e l'ottimizzazione del plugin tramite l'utilizzo di una cache locale a Kong.\\

Ovviamente è stato necessario acquisire delle competenze di base prima della realizzazione del progetto, come ad esempio:
Acquisire una conoscenza sufficiente del sistema di Version Control \emph{Git}.

Le competenze necessarie per lo svolgimento del progetto sono:
\begin{itemize}
	\item[$\bullet$]Conoscenza della metodologia di sviluppo Agile (acquisita durante lo stage).
	\item[$\bullet$]Conoscenza del sistema di Version Control \emph{Git} (acquisita durante lo stage).
	\item[$\bullet$]Conoscenza del linguaggio Java e della JVM (Java Virtual Machine) per la realizzazione del microservizio (acquisita durante gli studi accademici).
	\item[$\bullet$]Conoscenza del framework \emph{Spring} da utilizzare nello sviluppo del microservizio.
	\item[$\bullet$]Conoscenza della struttura e del funzionamento di Kong Gateway (acquisita durante lo stage).
	\item[$\bullet$]Conoscenza del linguaggio Lua per lo sviluppo del plugin (acquisita durante lo stage).
	\item[$\bullet$]Conoscenza degli strumenti per effettuare testing sul prodotto finito (acquisita durante lo stage).
\end{itemize}

\subsection{Ripartizione del lavoro svolto}
Lo stage aziendale ha avuto una durata di 225 ore, come previsto dal piano di studio Universitario per la Facoltà di Informatica. All’interno dell’azienda le ore sono state suddivise in cinque settimane lavorative full time, dal lunedì al venerdì dalle 09:00 alle 18:00.\\
Di seguito due tabelle che riassumono la ripartizione settimanale e oraria dello sviluppo del progetto.\\
% \begin{table}[]
% 	\centering
% 	\begin{tabular}{ |c|p{c}
% 		\hline
% 		\textbf{Settimana} & \textbf{Attività svolte} \\ \hline
% 		\multirow{3}{*}{Prima settimana} & Esposizione delle specifiche del progetto.  \\
% 		& Acquisizione delle competenze preliminari necessarie quali \emph{Git} e metodologia di sviluppo \emph{Agile}. \\ \hline
% 		\multirow{2}{*}{Seconda settimana} & Studio del framework \emph{Spring}.  \\
% 		& Sviluppo del microservizio. \\ \hline
% 		\multirow{1}{*}{Terza settimana} & Sviluppo del microservizio. \\ \hline
% 		\multirow{3}{*}{Quarta settimana} & Testing del microservizio. \\
% 		& Studio del funzionamento di Kong Gateway. \\
% 		& Configurazione Kong Gateway. \\ \hline
% 		\multirow{4}{*}{Quinta settimana} & Studio delle specifiche del plugin. \\
% 		& Studio della sintassi e della semantica del linguaggio \emph{Lua}. \\
% 		& Sviluppo del plugin per Kong Gateway. \\
% 		& Testing del prodotto finito. \\ \hline
% 	\end{tabular}
% 	\caption{Ripartizione settimanale del lavoro svolto}
% \end{table}
\begin{table}[]
	\centering
	\begin{tabular}{|c|c|}
	\hline
	\textbf{Settimana} & \textbf{Attività svolte} \\ \hline
	Prima settimana & \begin{tabular}[c]{@{}c@{}}Esposizione delle specifiche del progetto.\\ Acquisizione delle competenze preliminari necessarie\\ quali Git e metodologia di sviluppo Agile.\end{tabular} \\ \hline
	Seconda settimana & \begin{tabular}[c]{@{}c@{}}Studio del framework Spring.\\ Sviluppo del microservizio.\end{tabular} \\ \hline
	Terza settimana & Sviluppo del microservizio. \\ \hline
	Quarta settimana & \begin{tabular}[c]{@{}c@{}}Testing del microservizio.\\ Studio del funzionamento di Kong Gateway.\\ Configurazione Kong Gateway.\end{tabular} \\ \hline
	Quinta settimana & \begin{tabular}[c]{@{}c@{}}Studio delle specifiche del plugin.\\ Studio della sintassi e della semantica del linguaggio Lua.\\ Sviluppo del plugin per Kong Gateway.\\ Testing del prodotto finito.\end{tabular} \\ \hline
	\end{tabular}
	\caption{Ripartizione settimanale del lavoro svolto}
\end{table}

\begin{table}[]
	\centering
	\begin{tabular}{|c|c|}
	\hline
	\textbf{Attività svolte} & \textbf{Ore impiegate} \\ \hline
	Esposizione delle specifiche del progetto & 2 \\ \hline
	\begin{tabular}[c]{@{}c@{}}Acquisizione delle competenze preliminari \\ necessarie quali Git e metodologia di sviluppo Agile\end{tabular} & 38 \\ \hline
	Studio del framework Spring & 20 \\ \hline
	Sviluppo del microservizio & 60 \\ \hline
	Testing del microservizio & 5 \\ \hline
	Studio del funzionamento di Kong Gateway & 30 \\ \hline
	Configurazione Kong Gateway & 5 \\ \hline
	Studio delle specifiche del plugin & 2 \\ \hline
	Studio della sintassi e della semantica del linguaggio Lua & 12 \\ \hline
	Sviluppo plugin per Kong Gateway & 20 \\ \hline
	Testing del prodotto finito & 8 \\ \hline
	\end{tabular}
	\caption{Ripartizione oraria del lavoro svolto}
\end{table}

%TROVA MODO DI SPAZIARE VERTICALMENTE LE CELLE

%\section{Introduzione al progetto}\label{sec:introduzione}
\section{Obiettivi}\label{sec:obiettivi}